\documentclass{ximera}
\outcome{Awareness of correspondence between definite integral and area under a curve.}
\outcome{Familiarity with integral notation.}
\outcome{Familiarity with Riemann sum notation.}
\title{Introduction to definite integrals}
\author{Alexander Taam}



\begin{document}
\begin{abstract}
  We motivate, and develop the foundations of, integration.
\end{abstract}
\maketitle

When motivating our study of instantaneous rates of change and differentiation, we considered the problem of calculating (instantaneous) velocity from a given position function. Now we turn our attention to (almost) the opposite problem. If the velocity function of a moving object is given, is it possible determine the position function of the object, or at least calculate the displacement (i.e. change in position) of the object over a given interval?

For simple enough velocity functions, specifically constant functions $v(t)=c$, this is straightforward. If an object traveling at a constant rate of $c$ meters per second over the time interval $[a,b]$, then $c=\frac{s(d)-s(c)}{d-c}$ for any subinterval $[c,d]$ of $[a,b]$ where $s(t)$ is the position in meters at time $t$ seconds (of course different choices for units of distance and time may be used as well). In other words the instantaneous velocity of the object at any time $t$, $a<t<b$ is equal to the average velocity over any subinterval of $[a,b]$.

\begin{example}[Displacement from constant instantaneous velocity]
Suppose a car on the highway drives North at $60$ mph for $5$ hours. How far is it to the location of the car after driving for $4$ hours and $15$ minutes, from the location of the car after $1$ hour? \[\answer{195}\]

\begin{feedback} Let $s(t)$ be the distance of the car, in miles, North of its starting point, after driving for $t$ hours. The car travels $60$ miles (in the same direction) each of the three hours: from time $t=1$ to $t=2$, then from $t=2$ to $t=3$, from $t=3$ to $t=4$. Finally, in the next quarter hour, from $t=4$ to $t=4.25$, the car travels one quarter of $60$ miles. So in total the car travels $60+60+60+15=3.25\cdot60=s(4.25)-s(1)$, which is exactly the result of solving the equation $60=\frac{s(4.25)-s(1)}{4.25-1}$ for $s(4.25)-s(1)$.
\end{feedback}
\end{example}

If we graph the constant velocity function $y=v(t)=60$ from the previous example, notice that the displacement over the interval $[1,4.25]$ exactly corresponds to the area of the region between the lines $y=60$, $y=0$, $x=1$ and $x=4.25$.
\[
\graph{0\leq y\leq 60,1\leq x\leq 4.25}
\]


\end{document}
