\documentclass{ximera}
\input{../preamble.tex}
\outcome{Awareness of correspondence between definite integral and area under a curve.}
\outcome{Familiarity with integral notation.}
\outcome{Familiarity with Riemann sum notation.}
\title{How to approximate area under a curve, using Riemann sums}
\author{Alexander Taam}



\begin{document}
\begin{abstract}
  We describe a systematic way of approximating area of a curved region, using rectangles.
\end{abstract}
\maketitle

To define and calculate area of arbitrary shapes, we use approximating rectangles and limits.

\begin{definition}\label{def:partition}
\begin{foldable}
\unfoldable{
In other words, a regular partition of $n$ subintervals is the result of cutting up $[a,b]$ into $n$ pieces of the same size, and labeling the new endpoints $x_0,x_1,\dots,x_n$ in order.} 
\begin{foldable}
\unfoldable{In other words, a \emph{partition} of a closer interval $[a,b]$ is a set of numbers $a=x_0<x_1<x_2<\dots<x_{n-1}<x_n=b$. It is a \emph{regular partition} if:} each $x_i-x_{i-1}=\frac{b-a}{n}$. In that case denote the value $\frac{b-a}{n}$ by $\Delta x$
\end{foldable}
\end{foldable}
\end{definition}

\end{document}
